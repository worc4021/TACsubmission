
\documentclass[11pt, oneside]{article}

\usepackage[a4paper, tmargin=30mm, bmargin=30mm]{geometry}
\usepackage{graphicx}
\usepackage{xcolor}
\usepackage{enumerate}	
\usepackage{amsmath}
\usepackage{amssymb,mathrsfs}
\usepackage{parskip}

\newcounter{thmcount}
\renewcommand{\thethmcount}{\arabic{thmcount}}

\newtheorem{thm}[thmcount]{Theorem}
\newtheorem{cor}[thmcount]{Corollary}
\newtheorem{prop}[thmcount]{Proposition}

\def\X{\mathcal{X}}
\def\U{\mathcal{U}}
\def\V{\mathcal{V}}
\def\W{\mathcal{W}}
\def\Y{\mathcal{Y}}

\def\E{\mathbb{E}}
\def\P{\mathbb{P}}
\def\R{\mathbb{R}}
\def\B{\mathscr{B}}
\begin{document}

Let $\|x\|=\sqrt{x^T x}$ and let $\|x\|_P=\sqrt{x^T P x}$ for a positive definite matrix $P$. \\
Denote the $P$-ball as $\mathscr B_P(r) = \{x: \|x\|_P \leq r\}$. \\
The Minkowski sum, $\mathcal{X}\oplus\mathcal{Y}$, of sets $\mathcal{X},\mathcal{Y}$ is the set $\{z: \exists (x,y)\in\mathcal{X}\times\mathcal{Y}, z = x+y\}$.

\begin{prop}
The two following equalities hold:
\begin{subequations}\label{eq:balladd}
\begin{equation}
  \mathscr B_P(r_1+r_2) = \mathscr B_P(r_1)\oplus\mathscr B_P(r_2)
\end{equation}
\begin{equation}
  \mathscr B_P(r_1-r_2) = \mathscr B_P(r_1)\ominus\mathscr B_P(r_2) .
\end{equation}
\end{subequations}
\end{prop}

\textit{Proof:}\hspace{1ex}
Let $x_1\in\B_P(r_1)$ and $x_2\in\B_P(r_2)$, then:
\begin{enumerate}[(i).]
\item
the triangle inequality implies $\|x_1+x_2\|_P\leq \|x_1\|_P+\|x_2\|_P$;
%and hence $\B(x_1)\oplus\B_P(x_2) \leq r_1 + r_2$;
\item
choose any $x_1$ such that $\|x_1\|_P = r_1$ and set $x_2 = (r_2/r_1)x_1$ then $\|x_1+x_2\|_P = (1+r_2/r_1)\|x_1\|_P = r_1 + r_2$.
\end{enumerate}
From (i) we have $\B_P(r_1)\oplus\B_P(r_2) \subseteq \B_P(r_1+r_2)$ but (ii) implies $\B_P(r_1)\oplus\B_P(r_2) \supseteq \B_P(r_1+r_2)$ and it follows that $\B_P(r_1) \oplus \B_P(r_2) = \B_P(r_1 + r_2)$.

Define $r_3 = r_1-r_2$ where $r_1\geq r_2 > 0$.
Then $\B_P(r_3 + r_2) = \B_P(r_3) \oplus \B_P(r_2)$ implies $\B_P(r_3) = \B_P(r_3 + r_2) \ominus \B_P(r_2)$, and hence $\B_P(r_1-r_2) = \B_P(r_1) \ominus \B_P(r_2)$.
\mbox{}\hfill$\square$



\end{document}  