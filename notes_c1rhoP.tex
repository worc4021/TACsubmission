
\documentclass[11pt, oneside]{article}

\usepackage[a4paper, tmargin=30mm, bmargin=30mm]{geometry}
\usepackage{graphicx}
\usepackage{xcolor}
\usepackage{enumerate}	
\usepackage{amsmath}
\usepackage{amssymb,mathrsfs}
\usepackage{parskip}

\newcounter{thmcount}
\renewcommand{\thethmcount}{\arabic{thmcount}}

\newtheorem{thm}[thmcount]{Theorem}
\newtheorem{cor}[thmcount]{Corollary}
\newtheorem{prop}[thmcount]{Proposition}

\def\X{\mathcal{X}}
\def\U{\mathcal{U}}
\def\V{\mathcal{V}}
\def\W{\mathcal{W}}
\def\Y{\mathcal{Y}}

\def\E{\mathbb{E}}
\def\P{\mathbb{P}}
\def\R{\mathbb{R}}
\def\B{\mathscr{B}}
\begin{document}

Define $c_1 = c_1(\varrho)$, $P = P(\varrho)$ and 
$\varrho = \arg\max_{r} (1-r) c_1(r)$, where $c_1(r) = 1/\sqrt{t(r)}$ and $t(r)$, $P(r)$ are defined for any given $r\in(\rho(\Psi),1)$ by
\[
\bigl(t(r),P(r)\bigr) 
= \arg\min_{t,P}\ t \ \ \text{subject to} \ \
r^2 P \succ \Psi^T P \Psi \ \ \text{and} \ \
\begin{bmatrix} t & \xi_j^T \\ \xi_j & P\end{bmatrix} \succ 0 \ \forall j\in\mathbb N_r ,
\]
which requires solution of a semidefinite program. The optimal value of $\varrho$ (and hence $c_1(\varrho)$ and $P(\varrho)$) can be determined by performing a line-search over $r\in(\rho(\Psi),1)$ to find the maximizer of $(1-r)c_1(r)$.
%; this requires the solution of a sequence of SDPs. 

\end{document}